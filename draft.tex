% Modified based on Xiaoming Sun's template

\documentclass{article}
    \usepackage{amsmath,amsfonts,amsthm,amssymb}
    \usepackage{setspace}
    \usepackage{fancyhdr}
    \usepackage{lastpage}
    \usepackage{extramarks}
    \usepackage{chngpage}
    \usepackage{soul,color}
    \usepackage{graphicx,float,wrapfig}
    \usepackage{ifpdf}
    \usepackage{geometry}

\geometry{a4paper,scale=0.8}
    
    \ifpdf
      \usepackage[pdftex,bookmarks,bookmarksopen,bookmarksdepth=3]{hyperref}
    \else
      \usepackage[pagebackref]{hyperref}  
    \fi
    
    % In case you need to adjust margins:
    % \topmargin=-0.45in      %
    % \evensidemargin=0in     %
    % \oddsidemargin=0in      %
    % \textwidth=6.5in        %
    % \textheight=9.0in       %
    % \headsep=0.25in         %
    
    % Setup the header and footer
    % \pagestyle{fancy}                                                       %
    % \chead{\Title}  %
    % \rhead{\firstxmark}                                                     %
    % \lfoot{\lastxmark}                                                      %
    % \cfoot{}                                                                %
    % \rfoot{Page\ \thepage\ of\ \protect\pageref{LastPage}}                          %
    % \renewcommand\headrulewidth{0.4pt}                                      %
    % \renewcommand\footrulewidth{0.4pt}                                      %
    
    % 可以自定义一些命令
    \newcommand{\Answer}{\ \\\textbf{Answer:} }
    \newcommand{\Acknowledgement}[1]{\ \\{\bf Acknowledgement:} #1}
    
    \newcommand\numberthis{\addtocounter{equation}{1}\tag{\theequation}}
    
    
    %%%%%%%%%%%%%%%%%%%%%%%%%%%%%%%%%%%%%%%%%%%%%%%%%%%%%%%%%%%%%
    
    
    %%%%%%%%%%%%%%%%%%%%%%%%%%%%%%%%%%%%%%%%%%%%%%%%%%%%%%%%%%%%%
    % 标题部分
    \title{\textmd{\bf Cryptography: Birthday Paradox}}
    \date{}
    \author{Yiheng Lin, Zhihao Jiang}
    %%%%%%%%%%%%%%%%%%%%%%%%%%%%%%%%%%%%%%%%%%%%%%%%%%%%%%%%%%%%%
    
    \begin{document}
    \begin{spacing}{1.1}
    \maketitle %\thispagestyle{empty}
    
    %%%%%%%%%%%%%%%%%%%%%%%%%%%%%%%%%%%%%%%%%%%%%%%%%%%%%%%%%%%%%
    % Begin edit from here
    
    \section*{1}
    \subsection*{1}
    Theorem 1.1. Let $S = \{1, 2, \cdots, N\}$. For n times, uniformly randomly draw one element from set S with replacement. Let $x_t$ be the element we draw at time t. Then $\forall p > 0$, there exists a constant $C_1$ such that when $n \geq C_1 \sqrt{N}$, we have
    $$Pr[\exists 1\leq i, j \leq n, i\not = j \text{ such that } x_i = x_j] > p$$.
    \begin{proof}
        Let X denote the event that $\exists i, j \leq t, i\not = j \text{ such that } x_i = x_j$, then we have
        \begin{equation}
            \begin{aligned}
                Pr[\bar{X}] &= \frac{N(N-1)\cdots (N-n+1)}{N^n}\\
                &= \prod_{i=1}^{n-1}(1 - \frac{i}{N})\\
                &\leq \prod_{i=1}^{n-1} exp(-\frac{i}{N})\\
                &= exp(-\frac{n(n-1)}{2N})
            \end{aligned}
        \end{equation}
        Let $C_1 = \sqrt{-2ln(1 - p)} + 1$. Then when $n \geq C_1\sqrt{N}$, we have
        \begin{equation}
            \begin{aligned}
                n(n-1) > (1 + \sqrt{-2ln(1 - p)\cdot N})(\sqrt{-2ln(1 - p)\cdot N}) > 2ln(2)\cdot N
            \end{aligned}
        \end{equation}
        Which is equivalent to $-\frac{n(n-1)}{2N} < ln(1 - p)$.
        Thus use (1) we have
        $$Pr[\bar{X}] \leq exp(-\frac{n(n-1)}{2N}) < 1 - p$$
        So we have
        $$Pr[X] > p$$
    \end{proof}
    \subsection*{2}
    Lemma 1.1. For positive integer $n < N$, we have
    $$\sum_{i=1}^{n-1}ln(1 - \frac{i}{N}) >-\frac{n^2}{N}$$
    \begin{proof}
        Notice that $\forall x\in [1 -\frac{i+1}{N}, 1 - \frac{i}{N}]$ $(0\leq i\leq n)$, we have $ln(x) < ln(1 - \frac{i}{N})$. Thus
        $$\frac{1}{N}ln(1 - \frac{i}{N}) \geq \int_{1 - \frac{i+1}{N}}^{1 - \frac{i}{N}}ln(x)dx$$
        Thus we have
        \begin{equation}
            \begin{aligned}
                \frac{1}{N}\sum_{i=1}^{n-1}ln(1 - \frac{i}{N}) &\geq \int_{1-\frac{n}{N}}^1 ln(x)dx\\
                &= (xln(x) - x)|_{1-\frac{n}{N}}^1\\
                &= -\frac{n}{N} - (1 - \frac{n}{N})ln(1 - \frac{n}{N})\\
                &> -\frac{n}{N} - (1 - \frac{n}{N})(-\frac{n}{N})\\
                &= - \frac{n^2}{N^2}
            \end{aligned}
        \end{equation}
        Thus
        $$\sum_{i=1}^{n-1}ln(1 - \frac{i}{N}) > -\frac{n^2}{N}$$
    \end{proof}
    Theorem 1.2. Let $S = \{1, 2, \cdots, N\}$. For n times, uniformly randomly draw one element from set S with replacement. Let $x_t$ be the element we draw at time t. Then $\forall p > 0$, there exists a constant $C_2$ such that when $n \leq C_2 \sqrt{N}$, we have
    $$Pr[\exists 1\leq i, j \leq n, i\not = j \text{ such that } x_i = x_j] < p$$.
    \begin{proof}
        Let X denote the event that $\exists i, j \leq t, i\not = j \text{ such that } x_i = x_j$.

        Use Lemma 1.1, we have
        \begin{equation}
            \begin{aligned}
                Pr[\bar{X}] &= \frac{N(N-1)\cdots (N-n+1)}{N^n}\\
                &= \prod_{i=1}^{n-1}(1 - \frac{i}{N})\\
                &= exp(\sum_{i=1}^{n-1}ln(1 - \frac{i}{N}))\\
                &> exp(-\frac{n^2}{N})
            \end{aligned}
        \end{equation}
        Let $C_2 = \sqrt{-ln(1-p)}$. Then when $n \leq C_2\sqrt{N}$, we have
        $$exp(-\frac{n^2}{N}) \geq 1 - p$$
        So $Pr[\bar{X}] > 1 - p$, thus
        $$Pr[X] < p$$
    \end{proof}
    \section*{2}
    Theorem 2.1. Let $S = \{1, 2, \cdots, N\}$. Let $D_1: S\to R^+\cup \{0\}$ be a discrete probability distribution over S. For n times, randomly draw one element from set S according to distribution $D_1$ with replacement . Let $x_t$ be the element we draw at time t. Let $D_0$ be the uniform distribution over S, which satisfies $\forall i \in S, D_0(i) = \frac{1}{N}$. Then we have
    $$Pr_{D_1^n}[\exists 1\leq i, j \leq n, i\not = j \text{ such that } x_i = x_j] \geq Pr_{D_0^n}[\exists 1\leq i, j \leq n, i\not = j \text{ such that } x_i = x_j]$$.
    \begin{proof}
        Let X denote the event that $\exists i, j \leq t, i\not = j \text{ such that } x_i = x_j$. Let $X_m$ denote the event that $\exists 1\leq i, j \leq n, i\not = j \text{ such that } x_i = x_j = m$.

        First, to change $D_1$ to $D_0$, we can apply the following algorithm:
        \begin{enumerate}
            \item $t := 1$
            \item While $D_t \not = D_0$:
            \item $\quad$find $i, j \in S$ such that $D_t[i] < \frac{1}{N} < D_t[j]$
            \item $\quad$let $D_{t+1}[j] := D_t[i] + D_t[j] - \frac{1}{N}, D_{t+1}[i] := \frac{1}{N}, \forall k \not = i, j, D_{t+1}[k] := D_t[k]$
            \item $\quad t++$
            \item End While
        \end{enumerate}
        Since the number of $\frac{1}{N}$ in D increases at each iteration, this algorithm will terminate in N steps.

        We only need to prove that $$\forall t, Pr_{D_t^n}[X] \geq Pr_{D_{t+1}^n}[x]$$

        Without losing generality, suppose when generate $D_{t+1}$ from $D_t$, we choose $i = 1, j = 2$.

        Let $Y$ be the number of times that the element we draw is in $\{1, 2\}$.
        \begin{equation}
            \begin{aligned}
                Pr_{D^n}[X] &= Pr_{D^n}[\bigcup_{k = 3}^N X_k] + (1 - Pr_{D^n}[\bigcup_{k = 3}^N X_k])Pr_{D^n}(X_1\cup X_2 | \bigcap_{k = 3}^N \bar{X_k})\\
                &= Pr_{D^n}[\bigcup_{k = 3}^N X_k] + (1 - Pr_{D^n}[\bigcup_{k = 3}^N X_k])\sum_{i=0}^\infty Pr_{D^n}(Y = i | \bigcap_{k = 3}^N \bar{X_k}) Pr_{D^n}(X_1\cup X_2 | Y = i)
            \end{aligned}
        \end{equation}
        The last equation holds because $\forall i, Pr_{D^n}(X_1\cup X_2 | Y = i) = Pr_{D^n}(X_1\cup X_2 | Y = i, \bigcap_{k = 3}^N \bar{X_k})$.

        Notice that 
        \begin{equation}
            \begin{aligned}
                &\quad Pr_{D_t^n}(X_1\cup X_2 | Y = 2) - Pr_{D_{t+1}^n}(X_1\cup X_2 | Y = 2)\\
                &= \frac{1}{(D_t[1] + D_t[2])^2}(D_t[1]^2 + D_t[2]^2 - (\frac{1}{N})^2 - (D_t[1] + D_t[2] - \frac{1}{N})^2)\\
                &= -\frac{2}{(D_t[1] + D_t[2])^2}(D_t[1] - \frac{1}{N})(D_t[2] - \frac{1}{N})\\
                &> 0
            \end{aligned}
        \end{equation}
        And for any distribution D over S we have
        \begin{equation}
            Pr_{D^n}(X_1\cup X_2 | Y = i) =
            \begin{cases}
                0 & i = 0, 1\\
                1 & i \geq 3
            \end{cases}
        \end{equation}
        Thus
        $$\forall i, Pr_{D_t^n}(X_1\cup X_2 | Y = i) \geq Pr_{D_{t+1}^n}(X_1\cup X_2 | Y = i)$$
        Since we only adjust $D_t[0], D_t[1]$,
        $$Pr_{D_t^n}[\bigcup_{k = 3}^N X_k] = Pr_{D_{t+1}^n}[\bigcup_{k = 3}^N X_k]$$
        $$\forall i, Pr_{D_t^n}(Y = i | \bigcap_{k = 3}^N \bar{X_k}) = Pr_{D_{t+1}^n}(Y = i | \bigcap_{k = 3}^N \bar{X_k})$$
        So consider equation (6) and we get
        $$Pr_{D_t^n}[X] \geq Pr_{D_{t+1}^n}[x]$$
    \end{proof}
    \Acknowledgement{}
    
    % End edit to here
    %%%%%%%%%%%%%%%%%%%%%%%%%%%%%%%%%%%%%%%%%%%%%%%%%%%%%%%%%%%%%
    
    \end{spacing}
    \end{document}
    
    %%%%%%%%%%%%%%%%%%%%%%%%%%%%%%%%%%%%%%%%%%%%%%%%%%%%%%%%%%%%%
    