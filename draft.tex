% Modified based on Xiaoming Sun's template

\documentclass{article}
    \usepackage{amsmath,amsfonts,amsthm,amssymb}
    \usepackage{setspace}
    \usepackage{fancyhdr}
    \usepackage{lastpage}
    \usepackage{extramarks}
    \usepackage{chngpage}
    \usepackage{soul,color}
    \usepackage{graphicx,float,wrapfig}
    \usepackage{ifpdf}
    \usepackage{geometry}

\geometry{a4paper,scale=0.8}

    \ifpdf
      \usepackage[pdftex,bookmarks,bookmarksopen,bookmarksdepth=3]{hyperref}
    \else
      \usepackage[pagebackref]{hyperref}
    \fi

    % In case you need to adjust margins:
    % \topmargin=-0.45in      %
    % \evensidemargin=0in     %
    % \oddsidemargin=0in      %
    % \textwidth=6.5in        %
    % \textheight=9.0in       %
    % \headsep=0.25in         %

    % Setup the header and footer
    % \pagestyle{fancy}                                                       %
    % \chead{\Title}  %
    % \rhead{\firstxmark}                                                     %
    % \lfoot{\lastxmark}                                                      %
    % \cfoot{}                                                                %
    % \rfoot{Page\ \thepage\ of\ \protect\pageref{LastPage}}                          %
    % \renewcommand\headrulewidth{0.4pt}                                      %
    % \renewcommand\footrulewidth{0.4pt}                                      %

    % �����Զ���һЩ����
    \newcommand{\Answer}{\ \\\textbf{Answer:} }
    \newcommand{\Acknowledgement}[1]{\ \\{\bf Acknowledgement:} #1}
    \newtheorem{theorem}{Theorem}

    \newcommand\numberthis{\addtocounter{equation}{1}\tag{\theequation}}


    %%%%%%%%%%%%%%%%%%%%%%%%%%%%%%%%%%%%%%%%%%%%%%%%%%%%%%%%%%%%%


    %%%%%%%%%%%%%%%%%%%%%%%%%%%%%%%%%%%%%%%%%%%%%%%%%%%%%%%%%%%%%
    % ���ⲿ��
    \title{\textmd{\bf Cryptography: Birthday Paradox}}
    \date{}
    \author{Yiheng Lin, Zhihao Jiang}
    %%%%%%%%%%%%%%%%%%%%%%%%%%%%%%%%%%%%%%%%%%%%%%%%%%%%%%%%%%%%%

    \begin{document}
    \begin{spacing}{1.1}
    \maketitle %\thispagestyle{empty}

    %%%%%%%%%%%%%%%%%%%%%%%%%%%%%%%%%%%%%%%%%%%%%%%%%%%%%%%%%%%%%
    % Begin edit from here

    \section{}
    \subsection{}
    Theorem 1.1. Let $S = \{1, 2, \cdots, N\}$. For n times, uniformly randomly draw one element from set S with replacement. Let $x_t$ be the element we draw at time t. Then $\forall p > 0$, there exists a constant $C_1$ such that when $n \geq C_1 \sqrt{N}$, we have
    $$Pr[\exists 1\leq i, j \leq n, i\not = j \text{ such that } x_i = x_j] > p$$.
    \begin{proof}
        Let X denote the event that $\exists i, j \leq t, i\not = j \text{ such that } x_i = x_j$, then we have
        \begin{equation}
            \begin{aligned}
                Pr[\bar{X}] &= \frac{N(N-1)\cdots (N-n+1)}{N^n}\\
                &= \prod_{i=1}^{n-1}(1 - \frac{i}{N})\\
                &\leq \prod_{i=1}^{n-1} exp(-\frac{i}{N})\\
                &= exp(-\frac{n(n-1)}{2N})
            \end{aligned}
        \end{equation}
        Let $C_1 = \sqrt{-2ln(1 - p)} + 1$. Then when $n \geq C_1\sqrt{N}$, we have
        \begin{equation}
            \begin{aligned}
                n(n-1) > (1 + \sqrt{-2ln(1 - p)\cdot N})(\sqrt{-2ln(1 - p)\cdot N}) > 2ln(2)\cdot N
            \end{aligned}
        \end{equation}
        Which is equivalent to $-\frac{n(n-1)}{2N} < ln(1 - p)$.
        Thus use (1) we have
        $$Pr[\bar{X}] \leq exp(-\frac{n(n-1)}{2N}) < 1 - p$$
        So we have
        $$Pr[X] > p$$
    \end{proof}
    \subsection{}
    Lemma 1.1. For positive integer $n < N$, we have
    $$\sum_{i=1}^{n-1}ln(1 - \frac{i}{N}) >-\frac{n^2}{N}$$
    \begin{proof}
        Notice that $\forall x\in [1 -\frac{i+1}{N}, 1 - \frac{i}{N}]$ $(0\leq i\leq n)$, we have $ln(x) < ln(1 - \frac{i}{N})$. Thus
        $$\frac{1}{N}ln(1 - \frac{i}{N}) \geq \int_{1 - \frac{i+1}{N}}^{1 - \frac{i}{N}}ln(x)dx$$
        Thus we have
        \begin{equation}
            \begin{aligned}
                \frac{1}{N}\sum_{i=1}^{n-1}ln(1 - \frac{i}{N}) &\geq \int_{1-\frac{n}{N}}^1 ln(x)dx\\
                &= (xln(x) - x)|_{1-\frac{n}{N}}^1\\
                &= -\frac{n}{N} - (1 - \frac{n}{N})ln(1 - \frac{n}{N})\\
                &> -\frac{n}{N} - (1 - \frac{n}{N})(-\frac{n}{N})\\
                &= - \frac{n^2}{N^2}
            \end{aligned}
        \end{equation}
        Thus
        $$\sum_{i=1}^{n-1}ln(1 - \frac{i}{N}) > -\frac{n^2}{N}$$
    \end{proof}
    Theorem 1.2. Let $S = \{1, 2, \cdots, N\}$. For n times, uniformly randomly draw one element from set S with replacement. Let $x_t$ be the element we draw at time t. Then $\forall p > 0$, there exists a constant $C_2$ such that when $n \leq C_2 \sqrt{N}$, we have
    $$Pr[\exists 1\leq i, j \leq n, i\not = j \text{ such that } x_i = x_j] < p$$.
    \begin{proof}
        Let X denote the event that $\exists i, j \leq t, i\not = j \text{ such that } x_i = x_j$.

        Use Lemma 1.1, we have
        \begin{equation}
            \begin{aligned}
                Pr[\bar{X}] &= \frac{N(N-1)\cdots (N-n+1)}{N^n}\\
                &= \prod_{i=1}^{n-1}(1 - \frac{i}{N})\\
                &= exp(\sum_{i=1}^{n-1}ln(1 - \frac{i}{N}))\\
                &> exp(-\frac{n^2}{N})
            \end{aligned}
        \end{equation}
        Let $C_2 = \sqrt{-ln(1-p)}$. Then when $n \leq C_2\sqrt{N}$, we have
        $$exp(-\frac{n^2}{N}) \geq 1 - p$$
        So $Pr[\bar{X}] > 1 - p$, thus
        $$Pr[X] < p$$
    \end{proof}
    \section{}
    Theorem 2.1. Let $S = \{1, 2, \cdots, N\}$. Let $D_1: S\to R^+\cup \{0\}$ be a discrete probability distribution over S. For n times, randomly draw one element from set S according to distribution $D_1$ with replacement . Let $x_t$ be the element we draw at time t. Let $D_0$ be the uniform distribution over S, which satisfies $\forall i \in S, D_0(i) = \frac{1}{N}$. Then we have
    $$Pr_{D_1^n}[\exists 1\leq i, j \leq n, i\not = j \text{ such that } x_i = x_j] \geq Pr_{D_0^n}[\exists 1\leq i, j \leq n, i\not = j \text{ such that } x_i = x_j]$$.
    \begin{proof}
        Let X denote the event that $\exists i, j \leq t, i\not = j \text{ such that } x_i = x_j$. Let $X_m$ denote the event that $\exists 1\leq i, j \leq n, i\not = j \text{ such that } x_i = x_j = m$.

        First, to change $D_1$ to $D_0$, we can apply the following algorithm:
        \begin{enumerate}
            \item $t := 1$
            \item While $D_t \not = D_0$:
            \item $\quad$find $i, j \in S$ such that $D_t[i] < \frac{1}{N} < D_t[j]$
            \item $\quad$let $D_{t+1}[j] := D_t[i] + D_t[j] - \frac{1}{N}, D_{t+1}[i] := \frac{1}{N}, \forall k \not = i, j, D_{t+1}[k] := D_t[k]$
            \item $\quad t++$
            \item End While
        \end{enumerate}
        Since the number of $\frac{1}{N}$ in D increases at each iteration, this algorithm will terminate in N steps.

        We only need to prove that $$\forall t, Pr_{D_t^n}[X] \geq Pr_{D_{t+1}^n}[x]$$

        Without losing generality, suppose when generate $D_{t+1}$ from $D_t$, we choose $i = 1, j = 2$.

        Let $Y$ be the number of times that the element we draw is in $\{1, 2\}$.
        \begin{equation}
            \begin{aligned}
                Pr_{D^n}[X] &= Pr_{D^n}[\bigcup_{k = 3}^N X_k] + (1 - Pr_{D^n}[\bigcup_{k = 3}^N X_k])Pr_{D^n}(X_1\cup X_2 | \bigcap_{k = 3}^N \bar{X_k})\\
                &= Pr_{D^n}[\bigcup_{k = 3}^N X_k] + (1 - Pr_{D^n}[\bigcup_{k = 3}^N X_k])\sum_{i=0}^\infty Pr_{D^n}(Y = i | \bigcap_{k = 3}^N \bar{X_k}) Pr_{D^n}(X_1\cup X_2 | Y = i)
            \end{aligned}
        \end{equation}
        The last equation holds because $\forall i, Pr_{D^n}(X_1\cup X_2 | Y = i) = Pr_{D^n}(X_1\cup X_2 | Y = i, \bigcap_{k = 3}^N \bar{X_k})$.

        Notice that
        \begin{equation}
            \begin{aligned}
                &\quad Pr_{D_t^n}(X_1\cup X_2 | Y = 2) - Pr_{D_{t+1}^n}(X_1\cup X_2 | Y = 2)\\
                &= \frac{1}{(D_t[1] + D_t[2])^2}(D_t[1]^2 + D_t[2]^2 - (\frac{1}{N})^2 - (D_t[1] + D_t[2] - \frac{1}{N})^2)\\
                &= -\frac{2}{(D_t[1] + D_t[2])^2}(D_t[1] - \frac{1}{N})(D_t[2] - \frac{1}{N})\\
                &> 0
            \end{aligned}
        \end{equation}
        And for any distribution D over S we have
        \begin{equation}
            Pr_{D^n}(X_1\cup X_2 | Y = i) =
            \begin{cases}
                0 & i = 0, 1\\
                1 & i \geq 3
            \end{cases}
        \end{equation}
        Thus
        $$\forall i, Pr_{D_t^n}(X_1\cup X_2 | Y = i) \geq Pr_{D_{t+1}^n}(X_1\cup X_2 | Y = i)$$
        Since we only adjust $D_t[0], D_t[1]$,
        $$Pr_{D_t^n}[\bigcup_{k = 3}^N X_k] = Pr_{D_{t+1}^n}[\bigcup_{k = 3}^N X_k]$$
        $$\forall i, Pr_{D_t^n}(Y = i | \bigcap_{k = 3}^N \bar{X_k}) = Pr_{D_{t+1}^n}(Y = i | \bigcap_{k = 3}^N \bar{X_k})$$
        So consider equation (6) and we get
        $$Pr_{D_t^n}[X] \geq Pr_{D_{t+1}^n}[x]$$
    \end{proof}
    
    \section{}

    \subsection{}

    \begin{theorem}

    Let $S = \{1, 2, \cdots, N\}$. For $n$ times, uniformly randomly draw one element from set $S$ with replacement. Let $x_t$ be the element we draw at time $t$. Then for all integer $d\geq 2$ and for all $p > 0$, there exists a constant $C_1$ such that when $n \geq C_1 {N}^{\frac{d-1}{d}}$, we have
    $$Pr[X] > p,$$
    where $X$ denotes the event $\exists 1\leq i_1< i_2<\dots<i_d \leq n, \text{ such that } x_{i_1} = x_{i_2}=\dots=x_{i_d}$.

    \end{theorem}
    
    \begin{proof}
    
    We prove this theorem by induction. This theorem is right when $d=2$ which is proved before.
    
    The choice of $C_1$ is dependent of $p$ and $d$, we denote the constant as $C_1(p,d)$ in this proof.
    
    Now assume the theorem is right when $d=k-1$, and we prove the theorem is right when $d=k$.
    
    By induction, $\forall p$, we can find a constant $C_1$ such that
    $$Pr[X] > \frac{1+p}{2}$$
    for all N. Let $C_2$ be another constant such that $\frac{1+p}{2} \cdot (1 - (\frac{1}{e})^{C_2}) > p$ and $(1 - exp(-\frac{C_1 + C_2}{4})) > p$. We divide the whole drawing process into two steps:
    \begin{enumerate}
        \item First, draw $M_1 = C_1 \cdot N^{\frac{d-2}{d}} + C_2 \cdot N^{\frac{d-1}{d}}$ elements from S. Let set A be the set of all the elements that has been drawn for at least once.
        \item Second, draw $M_2 = 2(C_1 + C_2)\cdot N^{\frac{d-1}{d}}$ elements from S. Let $Y$ be the number of times that an element is drawn from set A.
    \end{enumerate}
    Now we consider 2 possible cases of the size of A.
    
    \subsubsection{First Case}
    If $|A| \leq n^{\frac{d-1}{d}}$:

    By assumption, after drawing for $C_1 \cdot N^{\frac{d-2}{d}}$ times, let event $E_1$ be that there exists an element $c_0$ in A that has been drawn for at least $d-1$ times, by assumption, given that $|A| \leq n^{\frac{d-1}{d}}$, we have $Pr(E_1) > \frac{p+1}{2}$.

    Notice that given a fixed element c in A, the probability that $|A|$ random draws draw c for 0 times is $(1 - \frac{1}{|A|})^{|A|} < \frac{1}{e}$. So let the event $E_2$ be that the element $c_0$ has been drawn for at least once in the last $C_2 \cdot N^{\frac{d-1}{d}}$ draws. Then we have the conditional probability $P(E_2|E_1) > (1 - (\frac{1}{e})^{C_2})$.

    If $E_1$ and $E_2$ both happens, then $c_0$ must be drawn for at least $(d-1) + 1 = d$ times. And the joint probability is
    \begin{equation}
        Pr(E_1, E_2) = Pr(E_1)\cdot Pr(E_2 | E_1) > \frac{p+1}{2} \cdot (1 - (\frac{1}{e})^{C_2}) > p
    \end{equation}
    Thus we have proved that $$Pr[X | |A| \leq N^{\frac{d-1}{d}}] > p$$ just after the first step.
    \subsubsection{Second Case}
    Else, we have $(C_1 + C_2)N^{\frac{d-1}{d}} > |A| > N^{\frac{d-1}{d}}$:

    Now we try to bound the probability that $Y < (|A|)^{\frac{d-2}{d-1}}$ in step 2.

    Let $X_i$ be the indicater random variable of whether the i th draw in step 2 draws an element in A. In other words,
    \begin{equation}
        X_i = \begin{cases}
            1 & \text{if the i th draw draws an element from A} \\
            0 & \text{otherwise}
        \end{cases}
    \end{equation}
    Then we have $Y = \sum_{i=1}^{M_2}X_i$. And $X_i$s are i.i.d. Bornoulli random variables. So use Chernoff Bound
    $$ Pr(X < (1 - \delta)\mu) \leq e^{-\frac{\delta^2 \mu}{2}}$$
    We have
    \begin{equation}
        Pr[Y \leq \frac{1}{2}E[Y]] \leq e^{-\frac{E[Y]}{8}}
    \end{equation}
    Here 
    $$E[Y] = \frac{|A|}{N} \cdot 2(C_1 + C_2)\cdot N^{\frac{d-1}{d}} \geq 2\cdot (|A|)^{\frac{d-2}{d-1}}$$
    And
    $$E[Y] = \frac{|A|}{N} \cdot 2(C_1 + C_2)\cdot N^{\frac{d-1}{d}} \geq 2\cdot (C_1 + C_2)\cdot N^{\frac{d-2}{d}}$$
    Thus we have
    $$Pr[Y > (|A|)^{\frac{d-2}{d-1}}] \geq 1 - exp(-\frac{(C_1 + C_2)\cdot N^{\frac{d-2}{d}}}{4}) \geq 1 - exp(-\frac{C_1 + C_2}{4})$$
    Let the event $E_3$ be that there exists an elemnet $c_1$ in A such that $c_1$ has been drawn for at least $d-1$ times in step 2. By induction, we know in step 2, the conditional probability
    $$Pr[E_3 | Y > (|A|)^{\frac{d-2}{d-1}}] > \frac{1+p}{2}$$
    Thus we have
    \begin{equation}
        Pr[E_3] = Pr[E_3 | Y > (|A|)^{\frac{d-2}{d-1}}]\cdot Pr[Y > (|A|)^{\frac{d-2}{d-1}}] > \frac{p+1}{2}\cdot (1 - exp(-\frac{C_1 + C_2}{4})) > p
    \end{equation}
    Since the event $E_3$ gaurantees that $c_1$ has been drawn for at least d times (at least (d-1) in step 2, and at least 1 in step 1), we proved that
    $$Pr[X | |A| > N^{\frac{d-1}{d}}] > p$$

    Combining subsection 1 and subsection 2, we get $Pr[X] > p$ for $d = k$. Thus we have finished the proof by induction.
    \end{proof}


    \subsection{}

    \begin{theorem}

    Let $S = \{1, 2, \cdots, N\}$. For $n$ times, uniformly randomly draw one element from set $S$ with replacement. Let $x_t$ be the element we draw at time $t$. Then for all integer $d\geq 2$ and for all $p > 0$, there exists a constant $C_2$ such that when $n \leq C_2 {N}^{\frac{d-1}{d}}$, we have
    $$Pr[X] < p,$$
    where $X$ denotes the event $\exists 1\leq i_1< i_2<\dots<i_d \leq n, \text{ such that } x_{i_1} = x_{i_2}=\dots=x_{i_d}$.

    \end{theorem}
    
    \begin{proof}
    
    Let $C_2=\sqrt[d]{p}$. We have
    \begin{align*}
    Pr[X]
    &\leq \sum_{i_1=1}^{C_2n}\sum_{i_2=i_1+1}^{C_2n}\dots\sum_{i_d=i_{d-1}+1}^{C_2n}Pr[x_{i_1} = x_{i_2}=\dots=x_{i_d}]  \\
    &=\sum_{i_1=1}^{C_2n}\sum_{i_2=i_1+1}^{C_2n}\dots\sum_{i_d=i_{d-1}+1}^{C_2n}\frac{1}{N^{d-1}}  \\
    &<\frac{C_2^dn^d}{N^{d-1}}  \\
    &=p.
    \end{align*}
    
    \end{proof}
    
    
    \Acknowledgement{}

    % End edit to here
    %%%%%%%%%%%%%%%%%%%%%%%%%%%%%%%%%%%%%%%%%%%%%%%%%%%%%%%%%%%%%

    \end{spacing}
    \end{document}

    %%%%%%%%%%%%%%%%%%%%%%%%%%%%%%%%%%%%%%%%%%%%%%%%%%%%%%%%%%%%%
    